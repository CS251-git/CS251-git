\documentclass{article}
\usepackage[utf8]{inputenc}

\title{git assignment}
\author{Anmol Porwal A.Faizan Kaartik Bhushan Prince Gaurav }
\date{April 2017}

\begin{document}

\maketitle

\section{3 Idiots}

\subsection{Introduction}
Two friends are searching for their long lost companion. They revisit their college days and recall the memories of their friend who inspired them to think differently, even as the rest of the world called them "idiots".

\subsection{Storyline}
Farhan Qureshi and Raju Rastogi want to re-unite with their fellow collegian, Rancho, after faking a stroke aboard an Air India plane, and excusing himself from his wife - trouser less - respectively. Enroute, they encounter another student, Chatur Ramalingam, now a successful businessman, who reminds them of a bet they had undertaken 10 years ago. The trio, while recollecting hilarious antics, including their run-ins with the Dean of Delhi's Imperial College of Engineering, Viru Sahastrabudhe, race to locate Rancho, at his last known address - little knowing the secret that was kept from them all this time

\section{Rang De basanti}
Rang De Basanti (IPA: [ˈrəŋɡ d̪eː bəˈsənt̪i]; English: Colour it Saffron) is a 2006 Indian drama film written, produced and directed by Rakeysh Omprakash Mehra. The literal meaning of the title can be translated as "Paint me with the colours of spring". It features an ensemble cast comprising Aamir Khan, Siddharth Narayan, Soha Ali Khan, Kunal Kapoor, R. Madhavan, Sharman Joshi, Atul Kulkarni and British actress Alice Patten in the lead roles. Made on a budget of ₹250 million (US dollars 3.7 million), it was shot in and around New Delhi. Upon release, the film broke all opening box office records in India. It was the highest-grossing film in its opening weekend in India and had the highest opening day collections for a Bollywood film. The film was well received and praised for strong screenplay and dialogues.
\end{document}

