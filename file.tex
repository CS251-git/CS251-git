\documentclass{article}
\usepackage[utf8]{inputenc}

\title{git assignment}
\author{Anmol Porwal A.Faizan Kaartik Bhushan Prince Gaurav }
\date{April 2017}

\begin{document}

\maketitle

\section{3 Idiots}

\subsection{Introduction}
Like so many Indian films, this is a very, very long film--with a run-time of almost three hours. When a movie is bad or just okay, this can seem like forever, but since "3 Idiots" is a very, very good film I loved its length. And, like most films of the genre, it has its share of the usual singing and dancing so foreign to films from other countries. One thing you should know, however, is that defining the type of film it is isn't really easy. Much of it is a comedy, but it also has many poignant moments (keep the Kleenex nearby), some existential moments where they explore the meaning of life and work and it's also a tender film about friendship. And, as my daughter pointed out when she saw the film, she loved that the men in the movie are not afraid to cry--something you rarely see in western films.

\section{Rang de Basanti}


\section{Rang De basanti}
Rang De Basanti (IPA: [ˈrəŋɡ d̪eː bəˈsənt̪i]; English: Colour it Saffron) is a 2006 Indian drama film written, produced and directed by Rakeysh Omprakash Mehra. The literal meaning of the title can be translated as "Paint me with the colours of spring". It features an ensemble cast comprising Aamir Khan, Siddharth Narayan, Soha Ali Khan, Kunal Kapoor, R. Madhavan, Sharman Joshi, Atul Kulkarni and British actress Alice Patten in the lead roles. Made on a budget of ₹250 million (US dollars 3.7 million), it was shot in and around New Delhi. Upon release, the film broke all opening box office records in India. It was the highest-grossing film in its opening weekend in India and had the highest opening day collections for a Bollywood film. The film was well received and praised for strong screenplay and dialogues.
\end{document}

